\documentclass[a4paper]{report}
\usepackage{setspace}
\pagestyle{plain}
\usepackage{amssymb,graphicx,color}
\usepackage{amsfonts}
\usepackage{latexsym}
\usepackage{amsmath}
\usepackage{hyperref}
\usepackage[backend=biber,style=numeric,sortcites,sorting=none]{biblatex}
\addbibresource{report.bib}
\usepackage[a4paper, margin = 3cm, bottom = 2.5cm]{geometry}

\newtheorem{theorem}{THEOREM}
\newtheorem{lemma}[theorem]{LEMMA}
\newtheorem{corollary}[theorem]{COROLLARY}
\newtheorem{proposition}[theorem]{PROPOSITION}
\newtheorem{remark}[theorem]{REMARK}
\newtheorem{definition}[theorem]{DEFINITION}
\newtheorem{fact}[theorem]{FACT}

\newtheorem{problem}[theorem]{PROBLEM}
\newtheorem{exercise}[theorem]{EXERCISE}
\def \set#1{\{#1\} }

\newenvironment{proof}{
PROOF:
\begin{quotation}}{
$\Box$ \end{quotation}}

\newcommand{\nats}{\mbox{\( \mathbb N \)}}
\newcommand{\rat}{\mbox{\(\mathbb Q\)}}
\newcommand{\rats}{\mbox{\(\mathbb Q\)}}
\newcommand{\reals}{\mbox{\(\mathbb R\)}}
\newcommand{\ints}{\mbox{\(\mathbb Z\)}}


\title{{\vspace{-14em}}
{{\Huge Leveraging FPGAs for building low-latency DNS Anomaly Detection and high-performance Firewall Solution}} \\
{\large}}

\date{Submission date: 27 March 2021}
\author{Patrick (Daiqi) Wu\thanks{
{\bf Disclaimer:}
This report is submitted as part requirement for Patrick (Daiqi) Wu's BSc Computer Science degree at UCL. It is
substantially the result of my own work except where explicitly indicated in the text.
The report will be distributed to the internal and external examiners, but thereafter may be copied and distributed under the Creative Commons -- Attribution 4.0 International License \cite{cc-by-4.0}.}
\\
BSc Computer Science\\ \\
Supervisors: Professor Stephen Hailes, Phil Demetriou}

\begin{document}
 
\onehalfspacing
\maketitle
\begin{abstract}
Summarise your report concisely.
\end{abstract}
\tableofcontents
\setcounter{page}{1}

\chapter{Introduction}

\section{Problem Statement}
The Domain Name System (DNS) Protocol is nowadays one of the most widely and prominently used protocols built-upon the Internet Protocol. As Akamai Technologies aggregated publicly available DNS data, the total amount of DNS traffic across the Internet quadrupled from $ 2 \times 10^{12}$ transactions/month (a query-reply pair) in January 2016, to $8 \times 10^{12}$ transactions/month in October 2020 \cite{DNS-Trends-and-Traffic}.


\section{Project Aim}

\section{Tools and Utilities}

\chapter{Background and Related Work}

\section{Background}

\subsection{Domain Name System (DNS)}
DNS is a distributed and hierarchical naming system for computing nodes or resources within a network \cite{RFC-1034}. It provides the backbone service of associating each entity's information with the domain names assigned to them \cite{RFC-1034, RFC-1035}. Notably on the Internet, DNS service is used when translating more readily memorised domain names to numerical IP addresses needed for locating servicing nodes in the network with the underlying IP Protocol \cite{RFC-1034, RFC-791}.

\subsection{Field Programmable Gate Arrays (FPGA)}

\subsection{Motivation}

\section{Related Work}

\chapter{Design}

\chapter{Simulation}

\chapter{Hardware Implementation}

\chapter{Hardware Debugging}

\chapter{Software Control System}

\chapter{Testing and Evaluation}

\chapter{Conclusions and Future Work}

\section{Conclusions}

\section{Future Work}
How the project might be continued, but don't give the impression you ran out of time!

\addcontentsline{toc}{chapter}{References}
\printbibliography[title=References]

\appendix

\chapter{Code}

\chapter{Demo}

\chapter{Project Proposal}

\end{document}